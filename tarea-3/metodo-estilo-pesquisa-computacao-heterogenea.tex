% Created 2025-10-01 Wed 00:21
% Intended LaTeX compiler: pdflatex
\documentclass[11pt]{article}
\usepackage[utf8]{inputenc}
\usepackage[T1]{fontenc}
\usepackage{graphicx}
\usepackage{longtable}
\usepackage{wrapfig}
\usepackage{rotating}
\usepackage[normalem]{ulem}
\usepackage{amsmath}
\usepackage{amssymb}
\usepackage{capt-of}
\usepackage{hyperref}
\author{Fabricio Puente Mansilla}
\date{\textit{<2025-09-30 Tue>}}
\title{Metodo de Pesquisa e Estilo de Pesquisa}
\hypersetup{
 pdfauthor={Fabricio Puente Mansilla},
 pdftitle={Metodo de Pesquisa e Estilo de Pesquisa},
 pdfkeywords={},
 pdfsubject={},
 pdfcreator={Emacs 30.1 (Org mode 9.7.11)}, 
 pdflang={English}}
\begin{document}

\maketitle
\tableofcontents

\section{Estilo de Pesquisa e Nível de Maturidade}
\label{sec:org1a3327f}

\subsection{Estilo de Pesquisa (Tipos de Pesquisa, Cap. 4.2)}
\label{sec:org89ae764}

O trabalho proposto deve adotar um estilo misto, combinando a Pesquisa
de Design com a Pesquisa Explicativa.  A Pesquisa de Design (Capítulo
8.5) é adequada porque o objetivo final é propor um novo artefato ou
método – neste caso, um conjunto de otimizações de compiler e runtime
que determinam "como as coisas poderiam ser".8 O artefato (o código
das otimizações) é a materialização do conhecimento gerado.  A
Pesquisa Explicativa (Capítulo 4.2) é essencial para garantir o rigor
científico. Não basta construir o artefato; é preciso explicar a causa
(a otimização introduzida no LLVM) que leva ao efeito (o aumento da
eficiência energética). A pesquisa explicativa busca os fatores
determinantes dos dados, indo além da simples descrição ou do
levantamento de opiniões.8 A relação de causalidade entre a
modificação do compiler e a mudança no consumo de Joules/GFLOPs é a
contribuição fundamental para a Ciência da Computação.
\subsection{Nível de Maturidade Científica (Cap. 5)}
\label{sec:orgea7a355}

Um trabalho de Mestrado exige uma contribuição significativa para o
avanço do conhecimento, não sendo suficiente a mera apresentação de um
produto ou protótipo O nível de maturidade mais apropriado para este
projeto é: Nível 5.4: Apresentação de Algo Reconhecidamente Melhor.
Este nível exige que a nova abordagem proposta seja comparada,
quantitativamente, com outras soluções existentes no estado da arte.
No contexto deste Mestrado, a abordagem proposta (LLVM otimizado) deve
ser confrontada com a baseline (LLVM/OpenMP padrão) usando métricas
objetivas de eficiência energética.  A adesão a este nível de
maturidade evita a armadilha do aluno que compara seu trabalho apenas
com suas versões anteriores, ou que escolhe uma ferramenta a priori.
Para avançar o estado da arte, o trabalho deve usar um teste-padrão
(ou um benchmark objetivo e replicável) e demonstrar que a solução
proposta apresenta vantagens claras em relação à solução dominante no
mercado ou na academia.
\section{Rigor Metodológico: Definições Operacionais e Variáveis}
\label{sec:org2dc2544}

O pilar do rigor científico é a objetividade, que se
materializa na definição clara e objetiva dos fenômenos a serem
observados. Para um sistema de low-power, termos como "melhor
desempenho" ou "maior eficiência" são subjetivos (definições
constitutivas) e devem ser transformados em Definições Operacionais.
\subsection{Variáveis de Pesquisa}
\label{sec:org5e1cd17}

O experimento de causalidade requer a identificação de variáveis
dependentes e independentes: Variável Independente (Manipulada): A
intervenção (otimização) aplicada ao pipeline de compilação
LLVM/OpenMP. Esta é uma variável Categórica Dicotômica, onde 0
representa a baseline e 1 representa a otimização proposta. Variáveis
Dependentes (Observadas): Latência de offload (ciclos de clock),
Desempenho (speedup) e, criticamente, Eficiência Energética.
\subsection{Definição Operacional da Eficiência Energética}
\label{sec:orgfe65481}

A métrica deve refletir o objetivo central da plataforma PULP, que é a
energia. A Eficiência Energética é operacionalmente definida pela
razão entre a energia total consumida (Joules, medida com ferramentas
de profiling no cluster RISC-V) e o trabalho útil realizado (Giga
Operações de Ponto Flutuante - GFLOPs).  Essa definição transforma o
constructo "ser eficiente" em uma Variável Contínua mensurável
(Joules/GFLOPs). Ao focar na energia total por operação útil, a
pesquisa garante que o ganho não seja apenas em velocidade, mas que a
otimização de latência realmente contribua para a premissa low-power
do hardware.
\end{document}
